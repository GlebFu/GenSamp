\documentclass[floatsintext,man]{apa6}

\usepackage{amssymb,amsmath}
\usepackage{ifxetex,ifluatex}
\usepackage{fixltx2e} % provides \textsubscript
\ifnum 0\ifxetex 1\fi\ifluatex 1\fi=0 % if pdftex
  \usepackage[T1]{fontenc}
  \usepackage[utf8]{inputenc}
\else % if luatex or xelatex
  \ifxetex
    \usepackage{mathspec}
    \usepackage{xltxtra,xunicode}
  \else
    \usepackage{fontspec}
  \fi
  \defaultfontfeatures{Mapping=tex-text,Scale=MatchLowercase}
  \newcommand{\euro}{€}
\fi
% use upquote if available, for straight quotes in verbatim environments
\IfFileExists{upquote.sty}{\usepackage{upquote}}{}
% use microtype if available
\IfFileExists{microtype.sty}{\usepackage{microtype}}{}

% Table formatting
\usepackage{longtable, booktabs}
\usepackage{lscape}
% \usepackage[counterclockwise]{rotating}   % Landscape page setup for large tables
\usepackage{multirow}		% Table styling
\usepackage{tabularx}		% Control Column width
\usepackage[flushleft]{threeparttable}	% Allows for three part tables with a specified notes section
\usepackage{threeparttablex}            % Lets threeparttable work with longtable

% Create new environments so endfloat can handle them
% \newenvironment{ltable}
%   {\begin{landscape}\begin{center}\begin{threeparttable}}
%   {\end{threeparttable}\end{center}\end{landscape}}

\newenvironment{lltable}
  {\begin{landscape}\begin{center}\begin{ThreePartTable}}
  {\end{ThreePartTable}\end{center}\end{landscape}}




% The following enables adjusting longtable caption width to table width
% Solution found at http://golatex.de/longtable-mit-caption-so-breit-wie-die-tabelle-t15767.html
\makeatletter
\newcommand\LastLTentrywidth{1em}
\newlength\longtablewidth
\setlength{\longtablewidth}{1in}
\newcommand\getlongtablewidth{%
 \begingroup
  \ifcsname LT@\roman{LT@tables}\endcsname
  \global\longtablewidth=0pt
  \renewcommand\LT@entry[2]{\global\advance\longtablewidth by ##2\relax\gdef\LastLTentrywidth{##2}}%
  \@nameuse{LT@\roman{LT@tables}}%
  \fi
\endgroup}


\ifxetex
  \usepackage[setpagesize=false, % page size defined by xetex
              unicode=false, % unicode breaks when used with xetex
              xetex]{hyperref}
\else
  \usepackage[unicode=true]{hyperref}
\fi
\hypersetup{breaklinks=true,
            pdfauthor={},
            pdftitle={Analysis Write-up},
            colorlinks=true,
            citecolor=blue,
            urlcolor=blue,
            linkcolor=black,
            pdfborder={0 0 0}}
\urlstyle{same}  % don't use monospace font for urls

\setlength{\parindent}{0pt}
%\setlength{\parskip}{0pt plus 0pt minus 0pt}

\setlength{\emergencystretch}{3em}  % prevent overfull lines


% Manuscript styling
\captionsetup{font=singlespacing,justification=justified}
\usepackage{csquotes}
\usepackage{upgreek}



\usepackage{tikz} % Variable definition to generate author note

% fix for \tightlist problem in pandoc 1.14
\providecommand{\tightlist}{%
  \setlength{\itemsep}{0pt}\setlength{\parskip}{0pt}}

% Essential manuscript parts
  \title{Analysis Write-up}

  \shorttitle{Analysis Write-up}


  \author{Gleb Furman\textsuperscript{1}}

  % \def\affdep{{""}}%
  % \def\affcity{{""}}%

  \affiliation{
    \vspace{0.5cm}
          \textsuperscript{1} Who Kneads a PH.D. Bakery  }



  




\usepackage{amsthm}
\newtheorem{theorem}{Theorem}[section]
\newtheorem{lemma}{Lemma}[section]
\theoremstyle{definition}
\newtheorem{definition}{Definition}[section]
\newtheorem{corollary}{Corollary}[section]
\newtheorem{proposition}{Proposition}[section]
\theoremstyle{definition}
\newtheorem{example}{Example}[section]
\theoremstyle{definition}
\newtheorem{exercise}{Exercise}[section]
\theoremstyle{remark}
\newtheorem*{remark}{Remark}
\newtheorem*{solution}{Solution}
\begin{document}

\maketitle

\setcounter{secnumdepth}{0}



\begin{table}[tbp]
\begin{center}
\begin{threeparttable}
\caption{\label{tab:tbl_desc}Descriptives of variables}
\begin{tabular}{lcccccclcccccclcccccclcccccclcccccclcccccclcccccc}
\toprule
 & \multicolumn{2}{c}{School} & \multicolumn{2}{c}{District Weighted} & \multicolumn{2}{c}{District Unweighted} \\
\cmidrule(r){2-3} \cmidrule(r){4-5} \cmidrule(r){6-7}
Variables & Mean & SD & Mean & SD & Mean & SD\\
\midrule
Number of Schools & NA & NA & 9,875.00 & 0.00 & 4.84 & 12.72\\
School Size & 579.07 & 203.19 & 534.77 & 225.34 & 534.77 & 225.34\\
Median Income & 60,084.98 & 25,007.61 & 56,710.63 & 20,804.82 & 56,648.44 & 20,750.06\\
Average Proportions &  &  &  &  &  & \\
\ \ \ ELA Proficiency & 0.59 & 0.23 & 0.60 & 0.20 & 0.60 & 0.20\\
\ \ \ Math Proficiency & 0.53 & 0.29 & 0.54 & 0.26 & 0.54 & 0.26\\
\ \ \ Economically Disadvantaged & 0.59 & 0.29 & 0.54 & 0.24 & 0.54 & 0.24\\
\ \ \ English Language Learners & 0.23 & 0.21 & 0.14 & 0.17 & 0.14 & 0.17\\
\ \ \ Minority Status & 0.65 & 0.30 & 0.46 & 0.32 & 0.46 & 0.32\\
\ \ \ Total/Free/Reduced Lunch & 0.59 & 0.29 & 0.53 & 0.24 & 0.53 & 0.23\\
Indicators &  &  &  &  &  & \\
\ \ \ Urban & 0.40 & 0.49 & 0.15 & 0.33 & 0.15 & 0.33\\
\ \ \ Suburban & 0.41 & 0.49 & 0.33 & 0.44 & 0.33 & 0.44\\
\ \ \ Town or Rural & 0.19 & 0.39 & 0.51 & 0.48 & 0.51 & 0.48\\
\bottomrule
\addlinespace
\end{tabular}
\begin{tablenotes}[para]
\textit{Note.} District variables are derived as aggregate means of school variables
\end{tablenotes}
\end{threeparttable}
\end{center}
\end{table}

\section{Cluster Analysis}\label{cluster-analysis}

\subsection{Population Frame}\label{population-frame}

The population frame is composed of data from three sources: (1) the
Common Core of Data (CCD), (2) publically available accountability data,
and (3) the U.S. Census. The CCD is a comprehensive database housing
anually collected national statistics of all public schools and
districts. Accountability data was used to calculate the proportion of
students within each school performing at or above proficiency in Math
and ELA. Finally, local median income was obtained from the U.S. Census
and was matched to each school by zipcode. School level data was
aggregated to get district level variables. These are reported in Table
@ref(tab:tbl\_desc)

\subsubsection{SUBS}\label{subs}

Stratification using balanced sampling (SUBS) was performed prior to
simulation because the group of schools in each strata would be static
across conditions except where the balancing model is manipulated. The
set of covariates in both the full model (SUBS-F) and the omitted
variable model (SUBS-OV) include binary indicator variables

\paragraph{Choosing the number of
clusters}\label{choosing-the-number-of-clusters}

\begin{itemize}
\tightlist
\item
  Everitt (2011), p126
\item
  Milligan and Cooper (1984)
\item
  list 30
\end{itemize}

\newpage

\section{References}\label{references}

\begingroup
\setlength{\parindent}{-0.5in} \setlength{\leftskip}{0.5in}

\hypertarget{refs}{}

\endgroup






\end{document}
