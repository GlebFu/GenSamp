\documentclass[floatsintext,man]{apa6}

\usepackage{amssymb,amsmath}
\usepackage{ifxetex,ifluatex}
\usepackage{fixltx2e} % provides \textsubscript
\ifnum 0\ifxetex 1\fi\ifluatex 1\fi=0 % if pdftex
  \usepackage[T1]{fontenc}
  \usepackage[utf8]{inputenc}
\else % if luatex or xelatex
  \ifxetex
    \usepackage{mathspec}
    \usepackage{xltxtra,xunicode}
  \else
    \usepackage{fontspec}
  \fi
  \defaultfontfeatures{Mapping=tex-text,Scale=MatchLowercase}
  \newcommand{\euro}{€}
\fi
% use upquote if available, for straight quotes in verbatim environments
\IfFileExists{upquote.sty}{\usepackage{upquote}}{}
% use microtype if available
\IfFileExists{microtype.sty}{\usepackage{microtype}}{}

% Table formatting
\usepackage{longtable, booktabs}
\usepackage{lscape}
% \usepackage[counterclockwise]{rotating}   % Landscape page setup for large tables
\usepackage{multirow}		% Table styling
\usepackage{tabularx}		% Control Column width
\usepackage[flushleft]{threeparttable}	% Allows for three part tables with a specified notes section
\usepackage{threeparttablex}            % Lets threeparttable work with longtable

% Create new environments so endfloat can handle them
% \newenvironment{ltable}
%   {\begin{landscape}\begin{center}\begin{threeparttable}}
%   {\end{threeparttable}\end{center}\end{landscape}}

\newenvironment{lltable}
  {\begin{landscape}\begin{center}\begin{ThreePartTable}}
  {\end{ThreePartTable}\end{center}\end{landscape}}




% The following enables adjusting longtable caption width to table width
% Solution found at http://golatex.de/longtable-mit-caption-so-breit-wie-die-tabelle-t15767.html
\makeatletter
\newcommand\LastLTentrywidth{1em}
\newlength\longtablewidth
\setlength{\longtablewidth}{1in}
\newcommand\getlongtablewidth{%
 \begingroup
  \ifcsname LT@\roman{LT@tables}\endcsname
  \global\longtablewidth=0pt
  \renewcommand\LT@entry[2]{\global\advance\longtablewidth by ##2\relax\gdef\LastLTentrywidth{##2}}%
  \@nameuse{LT@\roman{LT@tables}}%
  \fi
\endgroup}


  \usepackage{graphicx}
  \makeatletter
  \def\maxwidth{\ifdim\Gin@nat@width>\linewidth\linewidth\else\Gin@nat@width\fi}
  \def\maxheight{\ifdim\Gin@nat@height>\textheight\textheight\else\Gin@nat@height\fi}
  \makeatother
  % Scale images if necessary, so that they will not overflow the page
  % margins by default, and it is still possible to overwrite the defaults
  % using explicit options in \includegraphics[width, height, ...]{}
  \setkeys{Gin}{width=\maxwidth,height=\maxheight,keepaspectratio}
\ifxetex
  \usepackage[setpagesize=false, % page size defined by xetex
              unicode=false, % unicode breaks when used with xetex
              xetex]{hyperref}
\else
  \usepackage[unicode=true]{hyperref}
\fi
\hypersetup{breaklinks=true,
            pdfauthor={},
            pdftitle={Analysis Write-up},
            colorlinks=true,
            citecolor=blue,
            urlcolor=blue,
            linkcolor=black,
            pdfborder={0 0 0}}
\urlstyle{same}  % don't use monospace font for urls

\setlength{\parindent}{0pt}
%\setlength{\parskip}{0pt plus 0pt minus 0pt}

\setlength{\emergencystretch}{3em}  % prevent overfull lines


% Manuscript styling
\captionsetup{font=singlespacing,justification=justified}
\usepackage{csquotes}
\usepackage{upgreek}



\usepackage{tikz} % Variable definition to generate author note

% fix for \tightlist problem in pandoc 1.14
\providecommand{\tightlist}{%
  \setlength{\itemsep}{0pt}\setlength{\parskip}{0pt}}

% Essential manuscript parts
  \title{Analysis Write-up}

  \shorttitle{Analysis Write-up}


  \author{Gleb Furman\textsuperscript{1}}

  % \def\affdep{{""}}%
  % \def\affcity{{""}}%

  \affiliation{
    \vspace{0.5cm}
          \textsuperscript{1} Who Kneads a PH.D. Bakery  }



  




\usepackage{amsthm}
\newtheorem{theorem}{Theorem}[section]
\newtheorem{lemma}{Lemma}[section]
\theoremstyle{definition}
\newtheorem{definition}{Definition}[section]
\newtheorem{corollary}{Corollary}[section]
\newtheorem{proposition}{Proposition}[section]
\theoremstyle{definition}
\newtheorem{example}{Example}[section]
\theoremstyle{definition}
\newtheorem{exercise}{Exercise}[section]
\theoremstyle{remark}
\newtheorem*{remark}{Remark}
\newtheorem*{solution}{Solution}
\begin{document}

\maketitle

\setcounter{secnumdepth}{0}



\begin{table}[tbp]
\begin{center}
\begin{threeparttable}
\caption{\label{tab:tbl_desc}Descriptives of variables}
\begin{tabular}{lcccccclcccccclcccccclcccccclcccccclcccccclcccccc}
\toprule
 & \multicolumn{2}{c}{School} & \multicolumn{2}{c}{District Weighted} & \multicolumn{2}{c}{District Unweighted} \\
\cmidrule(r){2-3} \cmidrule(r){4-5} \cmidrule(r){6-7}
Variables & Mean & SD & Mean & SD & Mean & SD\\
\midrule
Number of Schools & NA & NA & 9,875.00 & 0.00 & 4.84 & 12.72\\
School Size & 579.07 & 203.19 & 534.77 & 225.34 & 534.77 & 225.34\\
Median Income & 60,084.98 & 25,007.61 & 56,710.63 & 20,804.82 & 56,648.44 & 20,750.06\\
Average Proportions &  &  &  &  &  & \\
\ \ \ ELA Proficiency & 0.59 & 0.23 & 0.60 & 0.20 & 0.60 & 0.20\\
\ \ \ Math Proficiency & 0.53 & 0.29 & 0.54 & 0.26 & 0.54 & 0.26\\
\ \ \ Economically Disadvantaged & 0.59 & 0.29 & 0.54 & 0.24 & 0.54 & 0.24\\
\ \ \ English Language Learners & 0.23 & 0.21 & 0.14 & 0.17 & 0.14 & 0.17\\
\ \ \ Minority Status & 0.65 & 0.30 & 0.46 & 0.32 & 0.46 & 0.32\\
\ \ \ Total/Free/Reduced Lunch & 0.59 & 0.29 & 0.53 & 0.24 & 0.53 & 0.23\\
Indicators &  &  &  &  &  & \\
\ \ \ Urban & 0.40 & 0.49 & 0.15 & 0.33 & 0.15 & 0.33\\
\ \ \ Suburban & 0.41 & 0.49 & 0.33 & 0.44 & 0.33 & 0.44\\
\ \ \ Town or Rural & 0.19 & 0.39 & 0.51 & 0.48 & 0.51 & 0.48\\
\bottomrule
\addlinespace
\end{tabular}
\begin{tablenotes}[para]
\textit{Note.} District variables are derived as aggregate means of school variables
\end{tablenotes}
\end{threeparttable}
\end{center}
\end{table}

\section{Cluster Analysis}\label{cluster-analysis}

\subsection{Population Frame}\label{population-frame}

The population frame is composed of data from three sources: (1) the
Common Core of Data (CCD), (2) publically available accountability data,
and (3) the U.S. Census. The CCD is a comprehensive database housing
anually collected national statistics of all public schools and
districts. Accountability data was used to calculate the proportion of
students within each school performing at or above proficiency in Math
and ELA. Finally, local median income was obtained from the U.S. Census
and was matched to each school by zipcode. School level data was
aggregated to get district level variables. These are reported in Table
@ref(tab:tbl\_desc)

\includegraphics{Method_files/figure-latex/distributions-1.pdf}
\includegraphics{Method_files/figure-latex/distributions-2.pdf}
\includegraphics{Method_files/figure-latex/distributions-3.pdf}
\includegraphics{Method_files/figure-latex/distributions-4.pdf}
\includegraphics{Method_files/figure-latex/distributions-5.pdf}

\subsection{SUBS}\label{subs}

Stratification using balanced sampling (SUBS) was performed prior to
simulation because the group of schools in each strata would be static
across conditions except where the balancing model is manipulated. The
set of covariates in both the full model (SUBS-F) and the omitted
variable model (SUBS-OV) include binary indicator variables

\subsubsection{Number of Clusters}\label{number-of-clusters}

Selecting the number of clusters, \(k\), is one of the most difficult
problems in cluster analysis (Steinley, 2006). To date, the most
extensive investigation of methods for determining \(k\) was conducted
by Milligan and Cooper (1985) who analyzed 30 methods. However, aside
from the limited generalizability of this study, many methods are also
innapropriate in the context of non-hierarchical and thus do not support
k-means clustering. Hennig and Liao (2013) argue that the method of
selecting \(k\) should depend on the context of the clustering and frame
the issue as one of obtaining an appropriate subject-matter-dependent
definition of rather than a statistical estimation.

\begin{itemize}
\item
  Everitt (2011), p126
\item
  clusterSim
\item
  Continuous data?

  \begin{itemize}
  \tightlist
  \item
    Calinski and Harabasz (1974)
  \item
    Duda and Hart (1973)
  \end{itemize}
\item
  Steinley, D. (2006a) K-means clustering: a half-century synthesis.
  British Journal of Mathematical \& Statistical Psychology, 59, 1--34.
\item
  Milligan and Cooper (1984)
\item
  list 30
\end{itemize}

\subsubsection{Subs-Full}\label{subs-full}

\begin{figure}
\centering
\includegraphics{Method_files/figure-latex/SUBS_FULL_CH-1.pdf}
\caption{}
\end{figure}

\begin{figure}
\centering
\includegraphics{Method_files/figure-latex/SUBS_FULL_SSB/SST-1.pdf}
\caption{}
\end{figure}

\subsubsection{Subs-OV}\label{subs-ov}

\begin{figure}
\centering
\includegraphics{Method_files/figure-latex/SUBS_OV_CH-1.pdf}
\caption{}
\end{figure}

\begin{figure}
\centering
\includegraphics{Method_files/figure-latex/SUBS_OV_SSB/SST-1.pdf}
\caption{}
\end{figure}

\subsubsection{6 Clusters}\label{clusters}

\begin{verbatim}
## # A tibble: 7 x 8
##   cluster_OV_6   `1`   `2`   `3`   `4`   `5`   `6`  `20`
##          <dbl> <int> <int> <int> <int> <int> <int> <int>
## 1           1.    NA     8  1332    NA    NA    NA  1340
## 2           2.    20    NA    NA    NA  1501    NA  1521
## 3           3.    NA    NA    NA   566    NA  1564  2130
## 4           4.     1  1019   228    NA    41    NA  1289
## 5           5.  1038    NA    NA   854    NA    NA  1892
## 6           6.   247  1453     3    NA    NA    NA  1703
## 7          20.  1306  2480  1563  1420  1542  1564    NA
\end{verbatim}

\begin{verbatim}
## # A tibble: 15 x 3
## # Groups:   cluster_full_6 [6]
##    cluster_full_6 cluster_OV_6     n
##             <int>        <int> <int>
##  1              6            3  1564
##  2              5            2  1501
##  3              2            6  1453
##  4              3            1  1332
##  5              1            5  1038
##  6              2            4  1019
##  7              4            5   854
##  8              4            3   566
##  9              1            6   247
## 10              3            4   228
## 11              5            4    41
## 12              1            2    20
## 13              2            1     8
## 14              3            6     3
## 15              1            4     1
\end{verbatim}

\begin{verbatim}
## Warning: attributes are not identical across measure variables;
## they will be dropped
\end{verbatim}

\begin{figure}
\centering
\includegraphics{Method_files/figure-latex/unnamed-chunk-2-1.pdf}
\caption{}
\end{figure}

\begin{verbatim}
## Warning: attributes are not identical across measure variables;
## they will be dropped
\end{verbatim}

\begin{figure}
\centering
\includegraphics{Method_files/figure-latex/unnamed-chunk-3-1.pdf}
\caption{}
\end{figure}

\subsubsection{10 Clusters}\label{clusters-1}

\begin{verbatim}
## # A tibble: 11 x 12
##    cluster_OV_10   `1`   `2`   `3`   `4`   `5`   `6`   `7`   `8`   `9`
##            <dbl> <int> <int> <int> <int> <int> <int> <int> <int> <int>
##  1            1.    NA  1218    NA    NA    NA    NA    NA    NA   153
##  2            2.    NA    NA     1    NA    NA   339    NA    NA    NA
##  3            3.  1104    NA    NA    NA    NA    NA    NA    NA    NA
##  4            4.    NA    NA   300    NA   136   220    NA    NA    NA
##  5            5.    NA    NA   357    NA   531    NA    NA    NA    NA
##  6            6.    50    NA    NA    NA    NA    NA    NA    NA   878
##  7            7.    NA    NA    NA     8    NA    NA     5  1452    NA
##  8            8.    NA    NA    NA  1130    NA    NA    57     3    NA
##  9            9.    NA    NA    NA    12     2    NA  1312     3    NA
## 10           10.    NA    23    NA    NA    NA    NA    NA    NA    37
## 11           20.  1154  1241   658  1150   669   559  1374  1458  1068
## # ... with 2 more variables: `10` <int>, `20` <int>
\end{verbatim}

\begin{verbatim}
## Warning: attributes are not identical across measure variables;
## they will be dropped
\end{verbatim}

\begin{figure}
\centering
\includegraphics{Method_files/figure-latex/unnamed-chunk-4-1.pdf}
\caption{}
\end{figure}

\begin{verbatim}
## Warning: attributes are not identical across measure variables;
## they will be dropped
\end{verbatim}

\begin{figure}
\centering
\includegraphics{Method_files/figure-latex/unnamed-chunk-5-1.pdf}
\caption{}
\end{figure}

\newpage

\section{References}\label{references}

\begingroup
\setlength{\parindent}{-0.5in} \setlength{\leftskip}{0.5in}

\hypertarget{refs}{}

\endgroup






\end{document}
